% LTex: language=pl
\section{Kontener wykonujący i ocenianie zadań}
Obecnie, ocena zadania polega na uruchomieniu wykonywalnego pliku wraz z podanym zestawem argumentów i porównaniu wyniku z oczekiwanym rezultatem. Jest to nieefektywne i potencjalnie niebezpieczne rozwiązanie. Lepszym rozwiązaniem jest odizolowanie procesu wykonania zadania do dedykowanego kontenera. Ów kontener może być uruchamiany w momencie potrzeby wykonania pliku wykonywalnego, co daje szersze możliwości jego konfiguracji albo działać w tle i nasłuchiwać na przychodzące pliki, co skróci proces wykonywania o czas uruchamiania kontenera. Szczególnie istotnym aspektem tego rozwiązania, jest możliwość unieszkodliwienia działania złośliwej aplikacji zawierające polecenia wyłączające system. Bezpieczniejszym, ale trudniejszym w implementacji rozwiązaniem jest uruchomienie dedykowanej maszyny wirtualnej, która całkowicie izolowałaby działanie szkodliwego oprogramowania. 
\newline \indent Ocena zadania polega na uruchomieniu programu \textit{judge.exe} dla podanych zestawów testów. Z racji braku parametryzacji programu, tworzenie innych strategii oceny zadania jest znacznie utrudnione.
\newline \indent W ramach pracy został przygotowany prototyp modułu wykonującego i oceniającego programy studentów. Znajduje się on w folderze \textit{/evaluator}. Ze względu na tworzone w ramach innej pracy inżynierskiej kontenery dedykowane do kompilacji różnych języków programowania, prototyp nie jest podłączony do ostatecznej wersji projektu i wymaga implementacji metod związanych z komunikacją z kontenerem. Proces oceny polega na odnalezieniu odpowiedniego działającego kontenera do wykonania zadania wraz z zestawem danych wejściowych, uruchomieniem go i ocenieniem rezultatu przy użyciu komponentu \textit{Grader}. Komponenty opierają się na implementacji interfejsów, co umożliwia na łatwiejszą rozbudowę aplikacji. Prototyp jest przystosowany zarówno do kontenera wykonującego pojedynczą komendę wykonania, jak i obsługującego cały zestaw testów. Posiada on różne algorytmy oceniania zadania, które mogą zostać rozszerzone poprzez nowe implementacje. Uzyskiwanie żądanego algorytmu odbywa się poprzez wzorzec fabryki, która zwraca instancję obiektu typu \textit{Grader}, na podstawie pożądanej strategii. Jako przykładowe algorytmy oceniania zostały zaimplementowane:
\begin{itemize}
    \item Domyślne ocenianie -- Porównuje wynik do oczekiwanego rezultatu i bierze pod uwagę limit czasowy wykonania. Stosowany w ramach oceny deterministycznych zadań.
    \item Ocenianie bez uwzględniania czasu -- Porównuje wynik do oczekiwanego rezultatu, bez uwzględnienia maksymalnego czasu wykonania. Używany, gdy docelowy program może działać przez długi okres czasu (na przykład programy nasłuchujące na zdarzenia).
    \item Ocenianie z dopuszczalnym błędem -- Ocenia wynik na podstawie jego zawierania w dopuszczalnym przedziale. Może być użyty w ocenie modeli sztucznej inteligencji.
\end{itemize}
Strategię oceniania oraz typ wykonywanego zadania można wybrać podczas uruchamiania prototypu. Rozwiązanie pozwala na zastosowanie systemu w ocenie kodu z większej ilości zajęć na Politechnice Gdańskiej.