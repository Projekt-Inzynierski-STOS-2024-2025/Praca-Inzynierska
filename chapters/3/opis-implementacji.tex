\section{Opis implementacji}
Zaimplementowany system został podzielony na dwa podsystemy, jeden z nich -- zarządca, komunikuje się z API istniejącego systemu STOS, a~drugi -- orkiestrator kontenerów, odpowiada za zlecanie pracy i odbieranie plików wynikowych z modułu kompilującego, czyli kontenerów odpowiedzialnych za kompilację i uruchamianie rozwiązań.

\subsection{Założenia stylistyczne implementowangeo kodu}
System został zaimplementowany w~języku Python. W~celu poprawy jakości, utrzymywalności, spójności i czytelności kodu oraz ułatwienia wykrywania błędów wszystkie metody oraz zmienne zostały oznaczone typami, a~klasy zostały utworzone na podstawie interfejsów zdefiniowanych przy użyciu modułu Abstract Base Classes(abc) \cite{pythonAbc}, pozwalającego na definiowanie klas abstrakcyjnych. Podczas implementacji zostało wykorzystane narzędzie Pyright służące do przeprowadzania statycznej analizy typów.

\subsection{Struktura kodu}
System posiada główny plik „main.py” umieszczony w~katalogu nadrzędnym repozytorium, w~którym następuje inicjalizacja obiektów klas zarządcy i~orkiestratora kontenerów, oraz zarejestrowanie w~tych obiektach wołań zwrotnych do siebie, odpowiedzialnych za przepływ danych między modułami systemu. Podsystemy zarządcy oraz orkiestratora wraz ze wszystkimi powiązanymi z~nimi komponentami zostały umieszczone w osobnych katalogach o nazwach „manager" dla katalogu zarządcy oraz „scheduler” dla katalogu orkiestratora.
\dirtree{%
.1 STOS.
.2 main.py.
.2 manager.
.3 manager.py.
.3 \_\_init\_\_.py.
.3 api\_diver.py.
.3 cache\_driver.py.
.3 storage\_driver.py.
.3 types.py.
.2 scheduler.
.3 scheduler.py.
.3 \_\_init\_\_.py.
}
Sposób funkcjonowania systemu, zawartość plików oraz szczegóły implementacyjne zostaną przedstawione w~kolejnych podrozdziałach poświęconych poszczególnym elementom systemu. 
