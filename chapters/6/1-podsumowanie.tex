Istniejący system został przeanalizowany i opisany w tej pracy, zarówno pod względem uruchamiania, jak i przepływu sterowania. Zostały przebadane możliwości użycia istniejących komponentów i ich skalowania. Moduł kompilujący oraz serwis STOS zostały zintegrowane z obecnym rozwiązaniem. Moduł kompilujący wymaga osobnego zestawu katalogów, w celu uniknięcia konfliktów między działającymi równolegle wieloma instancjami rywalizującymi o dane z nazwanego potoku, co zostało rozwiązane w docelowym systemie.
\newline \indent W celu zwiększenia niezawodności systemu, uruchomione kontenery są zarządzane przez naszą aplikację, która zapewnia stałą ilość działających kontenerów i ich wznawianie w przypadku awarii, aż do ilości, na którą pozwalają dostępne zasoby obliczeniowe. Decyzja o użyciu takiego rozwiązania zamiast komercyjnych platform konteneryzacji, takich jak Kubernetes albo Docker Swarm, wiązała się z założeniem działania zagnieżdżonego kontenera z modułem kompilującym. Logi generowane przez aplikację są przesyłane do platformy Elasticstack, będąca centralnym punktem gromadzącym i wizualizującym logi. Jako mechanizm sprawiedliwej oceny uniezależniony od obecnej ilości dostępnych zasobów obliczeniowych, sugerowane jest zliczanie niskopoziomowych operacji. Mechanizm ten nie został zaimplementowany, ze względu na skompilowanie i priorytet pozostałych aspektów.
\newline \indent Utworzony system jest przystosowany do obsługi plików pobieranych z serwisu STOS oraz do ich kompilacji. Aplikacja została przetestowana poprzez testy jednostkowe, testy manualne oraz uruchomioną, kilkudniową symulację. W celu zapewnienia jakości kodu zostało zintegrowane narzędzie Prospector, które służy do statycznej analizy kodu. By umożliwić prosty rozwój oprogramowania, rozwiązanie zostało oparte na udokumentowanych interfejsach, które zawierają gotowe implementacje. Dla modułów których funkcjonalności ze względów na brak plików źródłowych nie udało się odtworzyć, zostały utworzone symulacje, które w dalszych etapach rozwoju powinny zostać podmienione.
\newline \indent Rozwiązanie zostało umieszczone na serwerze politechniki. W tym celu zostały utworzone maszyny wirtualne na środowisko produkcyjne oraz deweloperskie, serwer logów oraz maszynę z narzędziem Xen Orchestra UI. Zastosowanie maszyn wirtualnych nadzorcy pierwszego stopnia xcp-ng które jest sprawdzonym, komercyjnym rozwiązaniem, umożliwia na proste zarządzanie i monitorowanie maszyn wirtualnych.
\newline \indent Obsługa systemu została zawarta spisana w plikach Markdown. Prezentują one sposoby konfiguracji i rozwoju systemu. Uruchamianie systemu zostało sprowadzone do uruchomienia pojedynczego skryptu.
\newline \indent Utworzony system spełnia większość określonych na początku założeń. Ze względu na późny termin otrzymania kodu źródłowego oraz braki pojedynczych plików, nie wszystkie funkcjonalności mogły zostać zaimplementowane. Do atutów rozwiązania możemy zaliczyć prostotę analizy działania i rozwoju, skalowalność oraz użycie nowoczesnych narzędzi. System po zaimplementowaniu brakujących modułów może stanowić udoskonaloną alternatywę dla platformy STOS, gotową do użycia w warunkach produkcyjnych.