% LTex: language=pl
\section{Organizacja pracy}
\indent Metodyką pracy jest Kanban. Do zarządzania przechowywaniem i przepływem zadań, używane jest narzędzie „Jira” stworzone i utrzymywane przez firmę Atlassian. W celu lepszego zrozumienia stanu poszczególnych implementowanych funkcjonalności zostały wprowadzone sekcje „Do zrobienia”, „W toku”, „Do weryfikacji”, „Gotowe”, „Wdrożone na produkcję”, do których każde z zadań jest przyporządkowane. Zadania są definiowane i umieszczane na tablicy po analizie wymagań dostarczonych przez interesariuszy oraz są uzupełnianie o kryteria akceptacji, opis i opcjonalny podział na pomniejsze zadania. Zostają one przypisane w momencie tworzenia, do osoby odpowiedzialnej, zgodnie z domeną specjalizacji w tworzonym projekcie.
\newline \indent Jako repozytorium kodu posłuży platforma GitHub. Poza przechowywaniem współtworzonego kodu umożliwia ona na wzajemną jego ocenę oraz stworzenie skryptów integrujących, testujących i wdrażających rozwiązania.
\newline \indent Jako baza wiedzy, zostanie użyte narzędzie Confluence, które podobnie jak Jira, jest stworzone i utrzymywanie przez firmę Atlassian. Będą tam zapisywane dokumenty związane z tworzonym systemem, szczególnie w zakresie jego uruchamiania i konfiguracji.
\newline \indent Do komunikacji pomiędzy zespołem, a interesariuszami, użyliśmy aplikacji Teams stworzonej przez firmę Microsoft, a do komunikacji pomiędzy zespołem, użyliśmy aplikacji Discord.

\section{Cele pracy}
\indent Praca ma zrealizować następujące cele:
\begin{itemize}
    \item Analiza działania aktualnego systemu, uwzględniając komunikację między serwisami, badania nad możliwością zrównolegleniem operacji, sposób działania kontenerów, wykrycie potencjalnych ulepszeń systemu oraz naszkicowanie diagramu przejść i wysokopoziomowej architektury.
    \item Zaproponowanie rozwiązań problemów z zakresu skalowania horyzontalnego systemu, gromadzenia i analizy logów oraz zapewnienia ciągłości działania.
    \item Implementacja środowiska realizującego aktualne zadania systemu, poprzez utworzenie i dostosowanie nowych komponentów oraz modyfikacji istniejących, zgodnie z dobrymi praktykami programistycznymi.
    \item Utworzenie maszyn wirtualnych, na których uruchomione będzie rozwiązanie oraz system logów, ze szczególnym uwzględnieniem aspektu prostoty ich zarządzania.
    \item Przygotowanie kompletnej instrukcji obsługi, instalacji oraz konfiguracji systemu.
\end{itemize}
