% LTex: language=pl
\chapter*{Streszczenie}
\indent System Testowania i Oceny Studentów, znany w formie akronimu jako STOS, jest systemem szeroko wykorzystywanym w ramach procesu dydaktycznego na Wydziale Elektroniki, Telekomunikacji i Informatyki na Politechnice Gdańskiej. Jest on fundamentalną częścią procesu automatycznej oceny zadań praktycznych zdawanych przez studentów, który pomaga prowadzącym w~szybkiej i dokładnej ewaluacji pracy studentów. 
\newline \noindent W tej pracy eksplorowany jest proces analizy dokumentacji i kodu źródłowego istniejącego systemu, ewolucja koncepcji dotyczących stworzenia rozwiązania i implementacja skalowalnej platformy serwerowej dla systemu STOS. Zaprezentowano wiele technologii szeroko stosowanych w rozwiązaniach przemysłowych, wraz z ich potencjalnym wykorzystaniem w kontekście istniejącego systemu STOS w celu nadania kontekstu decyzjom podjętym w procesie wdrażania i~implementacji platformy serwerowej. Szczegółowo opisane zostało zarówno rozwiązanie zaimplementowane w języku Python mające formę aplikacji serwerowej, jak i~konfiguracja środowiska wdrożenia zoptymalizowanego pod kątem wykorzystania wielu urządzeń, przekładając analizę teorytyczną wykorzystanych technologii na bardziej praktyczne realia właściwego wdrożenia. Na końcu, zawarto podsumowanie zaimplementowanego rozwiązania wraz z propozycjami dalszego rozwoju skalowalnej platformy dla systemu STOS.
\vspace{0.5cm}\newline
\textbf{Słowa kluczowe:} lorem ipsum, dolor sit amet, consectetur adipiscing\vspace{0.5cm}

\noindent \textbf{Dziedzina nauki i techniki, zgodnie z wymogami OECD:} nauki inżynieryjne i techniczne, robotyka i automatyka
