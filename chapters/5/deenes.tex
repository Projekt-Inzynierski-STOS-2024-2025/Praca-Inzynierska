% LTex: language=pl
\section{Serwis DNS (autor: Tymoteusz Paliński)}
Z racji na dość wysoki stopień skomplikowania architektury wdrożenia, w aktualnej fazie projektu dodawanie kolejnych maszyn i~usług zostało uznane za niewskazane -- zarówno przez ramy czasowe projektu, jak i~potencjalnie wysoki próg wejścia dla obecnego i~przyszłych administratorów systemu STOS-NEW. Nie oznacza to jednak, że architektura wdrożenia nie oferuje przestrzeni na dodanie komponentów usprawniających działanie i~zwiększających niezawodność platformy serwerowej dla systemu STOS-NEW.
\newline \noindent Jednym z~takich serwisów jest lokalny serwer DNS działający w~ramach klastra xcp-ng. W~aktualnej
implementacji platformy serwerowej komunikacja pomiędzy poszczególnymi maszynami odbywa się za pomocą bezpośredniej adresacji poprzez adresy IP każdej z maszyn. Podejście to cechuje się wysoką wrażliwością na zmiany konfiguracji sieciowej rozwiązania i~może skutkować wystąpieniem trudnych do wykrycia błędów podczas przyszłego rozwoju aplikacji. Dodanie własnego lokalnego serwera DNS pozwoliłoby na zniwelowanie tego problemu, jak i~potencjalne wprowadzenie dodatkowych mechanizmów zwiększających bezpieczeństwo systemu, takie jak wprowadzenie listy niedostępnych domen lub skorzystanie z mechanizmu \textit{round robin} w celu zwiększenia przepustowości ruchu sieciowego\cite{roundRobin}.
\newline \noindent Do wdrożenia takiego rozwiązania można wykorzystać gotowe oprogramowanie otwartoźródłowe, takie jak \textit{Pi-Hole} lub \textit{Unbound}\cite{pihole, unboundDns}. Obie z tych aplikacji oferują funkcjonalności lokalnego serwera DNS, aczkolwiek \textit{Unbound} charakteryzuje się większym stopniem optymalizacji w kontekście bezpieczeństwa sieci i~mechanizmu pamięci podręcznej zewnętrznych dla rekordów pochodzących z~serwerów zewnętrznych; z kolei \textit{Pi-Hole} cechuje się prostszą konfiguracją, lecz zazwyczaj używany jest w sieciach domowych jako prosty serwis \textit{cache} dla zewnętrznych serwerów DNS i wykorzystywany jest jako usługa \textit{AdBlock} blokująca reklamy w sieciach domowych. Zważając na ilość rozwiązań przemysłowych wykorzystanych w ramach architektury wdrożenia, serwis \textit{Unbound} wydaje się naturalnym wyborem dla wdrożenia go jako lokalnego serwera DNS w ramach klastra maszyn wirtualnych.

