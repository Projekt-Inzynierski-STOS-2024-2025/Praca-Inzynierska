\section{Proces uruchamiania}
Przed uruchomieniem systemu, należy zwrócić uwagę na konfigurację użytkowników oraz struktury katalogowej. Pliki uruchamianego systemu muszą być umieszczone w~katalogu \textit{/home/stos}, ze względu na niemodyfikowalne bez zmiany skryptów ustawienie ścieżek. Należy upewnić się, czy istnieją wszyscy użytkownicy, którzy są wymienieni w~skryptach, czyli „ja” oraz „stos”, oraz pozostałe skrypty i~pliki wykonywalne, używane w~trakcie działania modułów.
\newline \indent Uruchamianie platformy na docelowym serwerze odbywa się poprzez użycie dwóch skryptów:
\begin{itemize}
    \item start.sh -- uruchamianie kontenera kompilującego zadania.
    \item startup.sh -- uruchamianie serwisu pobierającego, przekazującego i~sprawdzającego zadania przesłane przez studentów.
\end{itemize}
Skrypt startup.sh pobiera adres VPN z interfejsu sieciowego \textit{tap1}. 

\section{Uproszczona struktura katalogowa}
\dirtree{%
.1 root.
.2 io-result.php.
.2 qapi.
.3 qctrl.php.
.2 stos.
.3 common-final.sh.
.3 fsapi.
.3 get.sh.
.3 handle-vc.sh.
.3 get.sh.
.3 judge.exe.
.3 qapi.
.3 runscript.
.3 sendresult.
.3 serve.sh.
.3 setup.sh.
.3 start.sh.
.3 startup.sh.
.3 stop.sh.
.3 sync.sh.
.3 vs-compile.sh.
.3 vc-final.sh.
.3 control.
.4 base.sh.
.4 regwine.reg.
.4 run.sh.
.4 vs.sh.
.4 vcbase.sh.
.3 io.
.4 result/.
.4 src/.
.4 input.
.4 output.
}
W zaprezentowanej strukturze katalogowej, ze względu na powtarzalność i~identyczność działania, nie zostały uwzględnione pliki kompilujące inne języki niż C lub C++. Pliki z folderu \textit{io}, służą do komunikacji z modułem kompilującym, a~skrypty z katalogu \textit{control}, są używane do jego sterowania. Część z plików jest w~postaci binarnej, przystosowanych do uruchomienia w~systemie Linux albo Windows.