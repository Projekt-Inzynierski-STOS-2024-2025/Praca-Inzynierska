% LTex: language=pl
\section{Dedykowane kontenery do kompilacji lub interpretacji większej ilości języków programowania (autor: Adam Niesiobędzki)}
Obecny system wykorzystywany jest głównie na zajęciach laboratoryjnych i~projektowych z~przedmiotów Podstawy Programowania oraz Algorytmy i~Struktury Danych, na których programowanie odbywa się z~wykorzystaniem języka programowania C++. Jego funkcjonalność mogłaby zostać rozszerzona poprzez implementację dedykowanych kontenerów przeznaczonych do kompilacji lub interpretacji większej ilości języków programowania, takich jak Python, C\# lub Java. Kod źródłowy rozwiązania byłby przekazywany do odpowiedniego kontenera odpowiedzialnego za kompilację lub interpretację na podstawie danych zadania pozyskanych z~zewnętrznego systemu STOS-ui, utworzony plik wykonywalny byłby następnie przetwarzany i~oceniany w taki sam sposób jak dzieje się to obecnie. Pozwoliłoby to na większą dowolność w wyborze technologii wykorzystanej do implementacji rozwiązania oraz wdrożenie systemu w proces kształcenia i~automatyzacji oceniania na zajęciach laboratoryjnych i~projektowych z~większej ilości przedmiotów. Kursy cechujące się oczekiwanym rezultatem wykonanego zadania w~obecnym toku nauczania kierunku Informatyka na Politechnice Gdańskiej, które mogłyby wykorzystać rozbudowany system to na przykład Programowanie Obiektowe, Metody Numeryczne, Platformy Technologiczne oraz Sztuczna Inteligencja.