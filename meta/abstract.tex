\chapter*{Abstract}
\indent STOS, or Student Testing and Grading System \textit{(System testowania i oceny studentów)} is a system used extensively as a part of the educational process at Faculty of Electronics, Telecommunications and Informatics at Gdańsk Tech. It acts as the backbone of the procedure of automatic grading of practical assignments submitted by students, which helps the course instructors with timely evaluation of the students' work.
\newline \noindent In this paper, we explore the process of analysis of the documentation and source code of the existing system, evolution of the concepts regarding the creation of a solution and finally, the implementation of a scalable server platform for the STOS system. A number of enterprise grade technologies and their use in context of the existing system is presented, in order to paint a clear picture of the choices made in relation to the implementation of the server platform. Both the solution implemented using Python in form of a backend application and configuration of a multi node server environment for said application are explored, embedding the theoretical analysis of used technologies in a more practical context. Finally, a brief summary of the solution is given, including possible future improvements to the server platform of the STOS system.
\vspace{0.5cm}\newline
\textbf{Keywords:} scalable server platform, distributed systems, cloud computing, containerization, virtualization \vspace{0.5cm}
